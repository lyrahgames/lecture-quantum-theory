\documentclass[10pt,fleqn]{article}
\usepackage{standard}
\usepackage{microtype}

\geometry{left=27mm,right=27mm,top=40mm,bottom=30mm}
\allowdisplaybreaks

\newcommand{\exerciseHeader}{%
  \hrule
  \begin{center}
    \Large
    \scshape
    Advanced Quantum Theory \\ Exercise Sheet 11
  \end{center}
  \medskip
  {
    \footnotesize
    \begin{minipage}[c]{0.49\textwidth}
      Markus Pawellek \\
      markuspawellek@gmail.com
    \end{minipage}
    \hfill
    \begin{minipage}[c]{0.49\textwidth}
      \raggedleft
      \today
    \end{minipage}
  }
  \medskip
  \hrule
  \bigskip
}

\newcommand{\bra}[1]{\left\langle #1 \right\vert}
\newcommand{\ket}[1]{\left\vert #1 \right\rangle}
\newcommand{\bracket}[2]{\left\langle #1 \middle\vert #2 \right\rangle}
\DeclareMathOperator{\derivative}{\mathrm{D}}

\begin{document}
  \selectlanguage{english}

  \exerciseHeader

  \section*{Problem 18: Perturbation Theory} % (fold)
  \label{sec:problem_18_perturbation_theory}
    \begin{multicols}{2}
      \paragraph{Preliminaries:}
      Let $\roundBrackets{\mathscr{H},\bracket{\cdot}{\cdot}}$ be a separable Hilbert space which is modeling the perturbed and the unperturbed system.
      For the formal definition of both systems we will define $\mathscr{I}$ as a countable index set with the same cardinality as an orthonormal basis of $\mathscr{H}$.

      \paragraph{Conditions of the Unperturbed System:}
      The Hamiltonian $\function{H}{\mathscr{H}}{\mathscr{H}}$ of the unperturbed system is assumed to have a discrete spectrum of real Energies.
      \[
        σ(H) = \set{E_n}{n\in\mathscr{I}}\subset\setReal
      \]
      For all $n\in\mathscr{I}$ we denote $\ket{n}\in\mathscr{H}$ to be the unique normalized eigenstate of $H$ with the eigenvalue $E_n$.
      \[
        H\ket{n} = E_n\ket{n}
        \separate
        \bracket{n}{n} = 1
      \]
      \[
        \forall m\in\mathscr{I},m\neq n: \quad E_m\neq E_n
      \]
      Based on this and due to the self-adjointness of the Hamiltonian $H$ we can derive that
      % , for all $n,m\in\mathscr{I}$ with $n\neq m$ we can derive that $\ket{m}$ and $\ket{n}$ are orthogonal.
      % \[
      %   \bra{m}H\ket{n} = E_n \bracket{m}{n} = E_m \bracket{m}{n}
      % \]
      % \[
      %   \implies (E_n-E_m) \bracket{m}{n} = 0
      % \]
      % \[
      %   \implies \bracket{m}{n} = 0
      % \]
      % Hence, for all $n,m\in\mathscr{I}$ we can state the following.
      % \[
      %   \bracket{m}{n} = δ_{mn}
      % \]
      $H$ itself is non-degenerate and that the set $\set{\ket{n}}{n\in\mathscr{I}}$ is building an orthonormal basis of the Hilbert space $\mathscr{H}$.

      \paragraph{Modelling the Perturbation:}
      Further, we describe the perturbation of the system by the self-adjoint operator $\function{V}{\mathscr{H}}{\mathscr{H}}$.
      For the application of $V$ to $H$ we will use an interaction strength parameter which is introduced by using a function $\tilde{H}$ as shown below.
      \[
        \function{\tilde{H}}{[0,1]}{\mathrm{L}(\mathscr{H},\mathscr{H})}
        \separate
        \tilde{H}(λ) \define H + λ V
      \]
      The continuous parameterization will enable us to analyze the solution of the perturbed system $\tilde{H}(λ)$ in form of a series expansion in terms of λ for all $λ\in[0,1]$.
      Please note that $\tilde{H}(1)$ is describing the system under the full perturbation.

      \paragraph{Conditions of the Perturbation:}
      We will assume that the contribution of $V$ is small and therefore not changes the essential properties.
      Hence, we choose $λ\in[0,1]$ to be arbitrary.
      $\tilde{H}(λ)$ has a discrete spectrum.
       % of non-degeneracy and that of a discrete spectrum of $\tilde{H}(λ)$ for all $λ\in[0,1]$.
      % Let $n\in\mathscr{I}$ be arbitrary.
      \[
        σ\roundBrackets{\tilde{H}(λ)} = \set{E_n(λ)}{n\in\mathscr{I}} \subset \setReal
      \]
      $\tilde{H}(λ)$ is non-degenerate and we again assume its eigenstates to be normalized and therefore to build an orthonormal eigenbasis of $\mathscr{H}$ with respect to $\tilde{H}(λ)$.
      So for all $n\in\mathscr{I}$ the following is true.
      \footnote{%
        The non-degeneracy of $\tilde{H}(λ)$ for all $λ\in[0,1]$ is not used explicitly.
        Therefore the derived formulas should be valid even if the perturbation described by $V$ introduces degeneracies into the non-degenerate system described by $H$.%
      }
      \footnote{%
        There is another normalization convention $\bracket{n}{n(λ)} = 1$ that we will discuss at the end.
      }
      \[
        \tilde{H}(λ) \ket{n(λ)} = E_n(λ) \ket{n(λ)}
      \]
      \[
        \forall m\in\mathscr{I},m\neq n: \quad E_m(λ)\neq E_n(λ)
      \]
      \[
        \bracket{n(λ)}{n(λ)} = 1
      \]
      Additionally, we assume that we are able to expand $E_n(λ)$ and $\ket{n(λ)}$ in terms of $λ$ for all $λ\in[0,1]$.
      Here the coefficients $E_{nk}$ are called the energy shifts and $\ket{n_k}$ are called the state shifts for all $n\in\mathscr{I}$ and $k\in\setNatural_0$.
      \[
        \set{E_{nk}}{}_{k=0}^\infty \subset \setReal
        \separate
        E_n(λ) = \sum_{k=0}^\infty λ^k E_{nk}
      \]
      \[
        \set{\ket{n_k}}{}_{k=0}^\infty\subset\mathscr{H}
        \separate
        \ket{n(λ)} = \sum_{k=0}^\infty λ^k \ket{n_k}
      \]

      \paragraph{Recursive Formulas for Correction Terms:}
      Taking the series expansions one can now derive inductive formulas for the coefficients.
      For that let $n\in\mathscr{I}$ and $λ\in[0,1]$ be arbitrary.
      We will start by computing the left-hand side of the Schrödinger equation for the perturbed system.
      \[
        \begin{aligned}
          \tilde{H}(λ)\ket{n(λ)}
          &= (H+λV)\sum_{k=0}^\infty λ^k \ket{n_k} \\
          &= \sum_{k=0}^\infty λ^k H\ket{n_k} + \sum_{k=0}^\infty λ^{k+1} V\ket{n_k} \\
          &= \sum_{k=0}^\infty λ^{k} H\ket{n_k} + \sum_{k=1}^\infty λ^{k} V\ket{n_{k-1}} \\
          &= H\ket{n_0} + \sum_{k=1}^\infty λ^{k} \roundBrackets{H\ket{n_k} + V\ket{n_{k-1}}}
        \end{aligned}
      \]
      For the right-hand side of the Schrödinger equation for the perturbed system we can do something similar.
      \[
        \begin{aligned}
          E_n(λ)\ket{n(λ)}
          &= \roundBrackets{\sum_{k=0}^\infty λ^k E_{nk}}\roundBrackets{\sum_{k=0}^\infty λ^k \ket{n_k}} \\
          &= \sum_{p,q=0}^\infty λ^{p+q} E_{np}\ket{n_q} \\
          &= \sum_{k=0}^\infty λ^k \sum_{p=0}^k E_{np}\ket{n_{k-p}}
        \end{aligned}
      \]
      Looking at the Schrödinger equation as a whole, we are now able to do a comparison of coefficients.
      This results in two equations.
      One for the starting values $E_{n0}$ and $\ket{n_0}$ and one recursive equation for all $k\in\setNatural$.
      \[
        H\ket{n_0} = E_{n0}\ket{n_0}
      \]
      \[
        H\ket{n_k} + V\ket{n_{k-1}} = \sum_{p=0}^k E_{np}\ket{n_{k-p}}
      \]

      \paragraph{Recursive Formulas of Normalization:}
      After inserting the series expansions into the Schrödinger equation we can do the same for the normalization condition.
      \[
        \begin{aligned}
          1 = \bracket{n(λ)}{n(λ)}
          &= \sum_{p,q=0} λ^{p+q} \bracket{n_p}{n_q} \\
          &= \sum_{k=0}^\infty λ^k \sum_{p=0}^k \bracket{n_p}{n_{k-p}}
        \end{aligned}
      \]
      Again we do comparison of coefficients and get two equations.
      One for the starting value $\bracket{n_0}{n_0}$ and one recursive equation for all $k\in\setNatural$.
      \[
        \bracket{n_0}{n_0} = 1
      \]
      \begin{align*}
        \bracket{n_0}{n_k} + \bracket{n_k}{n_0}
        &= 2 \Re(\bracket{n_0}{n_k}) \\
        &= -\sum_{p=1}^{k-1} \bracket{n_p}{n_{k-p}}
      \end{align*}
      To make the second equation simpler consider that the overall phase is not determined in quantum mechanics.
      Hence, without loss of generality, we may assume $\bracket{n_0}{n_k}=\Re(\bracket{n_0}{n_k})\in\setReal$ is purely real.
      For all $k\in\setNatural$ the second equation becomes the following.
      \[
        % \tcboxmath{
          \bracket{n}{n_k} = -\frac{1}{2}\sum_{p=1}^{k-1} \bracket{n_p}{n_{k-p}}
        % }
      \]

      \paragraph{Derivation of Energy Shifts:}
      After deriving these formulas we will first take a look at the starting value equations.
      \[
        H\ket{n_0} = E_{n0}\ket{n_0}
        \separate
        \bracket{n_0}{n_0} = 1
      \]
      We see that $\ket{n_0}$ is a normalized eigenfunction of $H$ with the eigenvalue $E_{n0}$.
      Therefore we can state that $E_{n0}$ lies in the spectrum of $H$ and $\ket{n_0}$ is the unique normalized eigenstate of $H$ with eigenvalue $E_{n0}$ and therefore belongs to the orthornormal basis.
      \[
        E_{n0} \in σ(H)
        \separate
        \ket{n_0} \in \set{\ket{m}}{m\in\mathscr{I}}
      \]
      We are free to choose the specific eigenstate because this is the freedom of permuting the eigenstates.
      But for consistency we will use the straightforward definition.
      \[
        \ket{n_0} \define \ket{n}
        \separate
        E_{n0} \define E_n
      \]
      Now we take the recursive equation for the energy and state shifts and insert the starting values to get an operator equation for $\ket{n_k}$ for all $k\in\setNatural$.
      \[
        \roundBrackets{H-E_n}\ket{n_k} = -V\ket{n_{k-1}} + \sum_{p=1}^{k} E_{np}\ket{n_{k-p}}
      \]
      But this equation gives us all information about the energy shift $E_{nk}$ as well.
      To see that we will apply $\bra{n}$ on the recursive equation for all $k\in\setNatural$ and use that $\ket{n}$ is an eigenstate of $H$ with eigenvalue $E_n$.
      \[
        \begin{aligned}
          0
          &= \bra{n}\roundBrackets{H-E_n}\ket{n_k} \\
          &= -\bra{n}V\ket{n_{k-1}} + \sum_{p=1}^k E_{np}\bracket{n}{n_{k-p}}
        \end{aligned}
      \]
      By solving the equation for $E_{nk}$ one gets the following explicit recursive formula for the energy shifts $E_{nk}$ for all $k\in\setNatural$.
      \[
        % \tcboxmath{
          E_{nk} = \bra{n}V\ket{n_{k-1}} - \sum_{p=1}^{k-1} E_{np}\bracket{n}{n_{k-p}}
        % }
      \]
      For this derivation we have only used the starting values of the normalization condition.
      These will be the same for both normalizations and therefore the explicit recursive formula for the energy shifts is also true for both normalizations.

      \paragraph{Derivation of State Shifts:}
      For the eigenstates $\ket{n_k}$ we basically have to invert the operator $H-E_n$.
      But due to its singularity this is not possible.
      For that reason we will first express $\ket{n_k}$ in terms of the orthonormal eigenbasis of $\mathscr{H}$ with respect to $H$.
      \[
        \begin{aligned}
          \ket{n_k}
          &= \sum_{m\in\mathscr{I}} \bracket{m}{n_k} \ket{m} \\
          &= \bracket{n}{n_k}\ket{n} + \sum_{\substack{m\in\mathscr{I}\\m\neq n}} \bracket{m}{n_k} \ket{m}
        \end{aligned}
      \]
      Let now $m\in\mathscr{I}$ with $m\neq n$ be arbitrary as well and apply $\bra{m}$ on the operator equation for $\ket{n_k}$ for all $k\in\setNatural$.
      \[
        \begin{aligned}
          \bra{m}&\roundBrackets{H-E_n}\ket{n_k}
          = \roundBrackets{E_m-E_n}\bracket{m}{n_k} \\
          &= -\bra{m}V\ket{n_{k-1}} + \sum_{p=1}^k E_{np}\bracket{m}{n_{k-p}} \\
          &= -\bra{m}V\ket{n_{k-1}} + \sum_{p=1}^{k-1} E_{np}\bracket{m}{n_{k-p}}
        \end{aligned}
      \]
      Solving this expression for $\bracket{m}{n_k}$ for all $k\in\setNatural$ gives us the following equations.
      \[
        \bracket{m}{n_k} = -\frac{\bra{m}V\ket{n_{k-1}}}{E_m-E_n} + \sum_{p=1}^{k-1} \frac{E_{np}\bracket{m}{n_{k-p}}}{E_m - E_n}
      \]
      Inserting this now into the expansion with respect to the eigenbasis yields the following nearly explicit recursive formula for $\ket{n_k}$ for all $k\in\setNatural$.
      \[
        % \tcboxmath{
          \begin{aligned}
            \ket{n_k} =
            &\bracket{n}{n_k}\ket{n} \\
            &+ \sum_{\substack{m\in\mathscr{I}\\m\neq n}} \bigg[
              \begin{aligned}[t]
                &-\frac{\bra{m}V\ket{n_{k-1}}}{E_m-E_n} \\
                &+ \sum_{p=1}^{k-1} \frac{E_{np}\bracket{m}{n_{k-p}}}{E_m - E_n} \bigg] \ket{m}
              \end{aligned}
          \end{aligned}
        % }
      \]
      To this moment we have not used the normalization condition for the derivation of $\ket{n_k}$.
      By using the recursive equation based on this condition we can then explicitly describe how to compute the state shifts.
      \[
          \begin{aligned}
            \ket{n_k} =
            &-\frac{1}{2}\sum_{p=1}^{k-1} \bracket{n_p}{n_{k-p}}\ket{n} \\
            &+ \sum_{\substack{m\in\mathscr{I}\\m\neq n}} \bigg[
              \begin{aligned}[t]
                &-\frac{\bra{m}V\ket{n_{k-1}}}{E_m-E_n} \\
                &+ \sum_{p=1}^{k-1} \frac{E_{np}\bracket{m}{n_{k-p}}}{E_m - E_n} \bigg] \ket{m}
              \end{aligned}
          \end{aligned}
      \]

      \paragraph{Computations to Second Order:}
      With this last equation we now have obtained explicit recursive formulas with their respective starting values for $E_{nk}$ and $\ket{n_k}$ for all $k\in\setNatural$.
      Through an iterative procedure beginning by $k=1$ one can now directly obtain the formulas for the energy and state shifts.
      \[
        E_{n1} = \bra{n}V\ket{n}
      \]
      \[
        \ket{n_1} = - \sum_{\substack{m\in\mathscr{I} \\ m\neq n}} \frac{\bra{m}V\ket{n}}{E_m-E_n} \ket{m}
      \]
      The same formulas will be used for $k=2$.
      But this time we will directly insert the equations for $E_{n1}$ and $\ket{n_1}$ obtained above.
      $E_{n2}$ can be computed directly.
      \[
        \begin{aligned}
          E_{n2}
          &= \bra{n}V\ket{n_1} - E_{n1}\bracket{n}{n_1} \\
          &= -\sum_{\substack{m\in\mathscr{I} \\ m\neq n}} \frac{\bra{m}V\ket{n}}{E_m-E_n} \bra{n}V\ket{m}
        \end{aligned}
      \]
      % For $\ket{n_2}$ we first use the recursive equation.
      \[
        \begin{aligned}
          \ket{n_2}
          &=
            \begin{aligned}[t]
              &-\frac{1}{2} \bracket{n_1}{n_{1}}\ket{n} \\
              &+ \sum_{\substack{m\in\mathscr{I}\\m\neq n}} \bigg[
                \begin{aligned}[t]
                  &-\frac{\bra{m}V\ket{n_{1}}}{E_m-E_n} \\
                  &+ \frac{E_{n1}\bracket{m}{n_{1}}}{E_m - E_n} \bigg] \ket{m}
                \end{aligned}
            \end{aligned}
        \end{aligned}
      \]
      % Now we insert the expressions for $E_{n1}$ and $\ket{n_1}$.
      \[
        \begin{aligned}
          = &-\frac{1}{2} \sum_{\substack{m,k\in\mathscr{I}\\m,k\neq n}} \frac{\bra{n}V\ket{m}}{E_m-E_n} \frac{\bra{k}V\ket{n}}{E_k-E_n} \bracket{m}{k} \ket{n} \\
          &+ \sum_{\substack{m\in\mathscr{I}\\m\neq n}} \Bigg[
            \begin{aligned}[t]
              & \sum_{\substack{k\in\mathscr{I} \\ k\neq n}} \frac{\bra{k}V\ket{n}}{E_k-E_n} \frac{\bra{m}V\ket{k}}{E_m-E_n}  \\
              & - \sum_{\substack{k\in\mathscr{I} \\ k\neq n}} \frac{\bra{k}V\ket{n}}{E_k-E_n} \frac{\bra{n}V\ket{n} \bracket{m}{k}}{E_m - E_n} \Bigg] \ket{m}
            \end{aligned}
          \\
          = &-\frac{1}{2} \sum_{\substack{m\in\mathscr{I}\\m\neq n}} \frac{\absolute{\bra{m}V\ket{n}}^2}{(E_m-E_n)^2} \ket{n} \\
          &+ \sum_{\substack{m,k\in\mathscr{I}\\m,k\neq n}} \frac{\bra{k}V\ket{n} \bra{m}V\ket{k}}{(E_k-E_n)(E_m-E_n)} \ket{m} \\
          &- \sum_{\substack{m\in\mathscr{I}\\m\neq n}} \frac{\bra{n}V\ket{n} \bra{m}V\ket{n}}{(E_m-E_n)^2} \ket{m}
        \end{aligned}
      \]

      \paragraph{Second Normalization:}
      We will now assume a different normalization condition.
      \[
        \bracket{n}{n(λ)} = 1
      \]
      Again, we insert the series expansion in terms of $λ$ and get the following.
      \[
        1 = \sum_{k=0}^\infty λ^k \bracket{n}{n_k} = \bracket{n}{n} + \sum_{k=1}^\infty λ^k \bracket{n}{n_k}
      \]
      Comparing by coefficients now gives us different and simpler result.
      \[
        \bracket{n}{n_0} = 1
        \separate
        \forall k\in\setNatural:\quad \bracket{n}{n_k} = 0
      \]
      With this the starting values for the recursive equations will not change.
      We insert the second equation into the recursive equations for the energy and state shifts at the time we have not used the normalization condition and get for all $k\in\setNatural$ the following.
      \[
        E_{nk} = \bra{n}V\ket{n_{k-1}}
      \]
      \[
        \begin{aligned}
          \ket{n_k} &= \sum_{\substack{m\in\mathscr{I}\\m\neq n}} \bigg[
            \begin{aligned}[t]
              & -\frac{\bra{m}V\ket{n_{k-1}}}{E_m-E_n} \\
              & + \sum_{p=1}^{k-1} \frac{E_{np}\bracket{m}{n_{k-p}}}{E_m - E_n} \bigg] \ket{m}
            \end{aligned}
        \end{aligned}
      \]
      In the second normalization $E_{n1}$ and $\ket{n_1}$ do not change.
      But for the second order correction we obtain the following.
      This will finish the proof.
      \[
        E_{n2} = \bra{n}V\ket{n_1}
      \]
      \[
        \begin{aligned}
          \ket{n_2}
          = &\sum_{\substack{m,k\in\mathscr{I}\\m,k\neq n}} \frac{\bra{k}V\ket{n} \bra{m}V\ket{k}}{(E_k-E_n)(E_m-E_n)} \ket{m} \\
          &- \sum_{\substack{m\in\mathscr{I}\\m\neq n}} \frac{\bra{n}V\ket{n} \bra{m}V\ket{n}}{(E_m-E_n)^2} \ket{m}
        \end{aligned}
      \]
      \hfill $\square$
    \end{multicols}
  % section problem_18_perturbation_theory (end)

  \newpage

  \section*{Problem 19: Scattering off a Radial Potential} % (fold)
  \label{sec:problem_19}
    \begin{multicols}{2}
      \paragraph{Preliminaries:}
      Let $\roundBrackets{\mathscr{H},\bracket{\cdot}{\cdot}}$ be the separable Hilbert space of three-dimensional square-integrable functions with Lebesgue-measure λ and the typical scalar product which models the scattering of a nonrelativistic particle of mass $m\in\setReal^+$.
      \[
        \mathscr{H}
        % \define \mathscr{L}^2\roundBrackets{\setReal^3}
        = \set{\function{φ}{\setReal^3}{\setComplex}}{\integral{\setReal^3}{}{\absolute{φ}^2}{λ} < \infty}
      \]
      \[
        \function{\bracket{\cdot}{\cdot}}{\mathscr{H}}{\mathscr{H}}
        \separate
        \bracket{φ}{ψ} \define \integral{\setReal^3}{}{\bar{φ}ψ}{λ}
      \]
      Let $V\in\mathscr{H}$ be the spherically symmetric potential of the given system.
      We will use the spherical standard parameterization $(M,μ)$ of $\setReal^3$.
      \[
        M\define \setReal^+\times[0,π]\times[0,2π)
      \]
      \[
        \function{μ}{M}{\setReal^3}
        \separate
        μ(r,ϑ,φ) \define
        \begin{pmatrix}
          r\sinϑ\cosφ \\
          r\sinϑ\sinφ \\
          r\cosϑ
        \end{pmatrix}
      \]
      The potential $V$ can be rewritten in form of a scaled radial potential $U$.
      \[
        \function{U}{\setReal^+}{\setReal}
      \]
      \[
        \forall x\in\setReal^3:\quad U(\norm{x}) \define \frac{2m}{\hbar^2}V(x)
      \]
      Hence, the typical Hamilton operator is given by the expression below.
      \[
        \function{H}{\mathscr{H}}{\mathscr{H}}
        \separate
        H \define -\frac{\hbar^2}{2m}\laplacian + V
      \]

      \paragraph{Expansion:}
      Let $k\in\setReal^+$ be arbitrary and let $\ket{ψ_k}\in\mathscr{H}$ be a solution of the Schrödinger equation with eigenvalue $E(k)$.
      \[
        H\ket{ψ_k} = E(k)\ket{ψ_k}
        \separate
        E(k)\define \frac{\hbar^2k^2}{2m}
      \]
      By inserting the definitions and multiplying with $2m\hbar^{-2}$ this equation takes on the following form.
      \[
        \roundBrackets{\laplacian - U\circ\norm{\cdot}  + k^2}\ket{ψ_k} = 0
      \]
      To model the scattering of a particle, we assume the particle to be a plane wave in $z$-direction.
      Because of that and due to the radial potential the wave function should not depend on the azimuth angle.
      So for all $(r,ϑ,φ)\in M$ we can reformulate the wave function as follows.
      \[
        \tilde{ψ}_k(r,ϑ) \define ψ_k\circ μ(r,ϑ,φ)
      \]
      Thanks to the spherically symmetric potential, we can expand $\ket{ψ_k}$ in terms of radial functions $\function{A_l}{\setReal^+}{\setComplex}$ and spherical harmonics $\mathscr{Y}_{l0}$ with coefficients $c_{l}\in\setComplex$ for all $l\in\setNatural_0$ and $r\in\setReal^+$.
      Therefore, we get the following for all $(r,ϑ,φ)\in M$.
      \begin{align*}
        \tilde{ψ}_k(r,ϑ)
        &= \sum_{l=0}^\infty c_{l} A_l(r) \mathscr{Y}_{l0}(ϑ) \\
        &= \sum_{l=0}^\infty \tilde{c}_l A_l(r) P_l(\cos ϑ)
      \end{align*}
      Here $P_l$ are the Legendre polynomials and $\tilde{c}_l\in\setComplex$ for all $l\in\setNatural_0$.
      For the expansion to be valid, the radial functions $A_l$ have to fulfill the radial Schrödinger equation for all $l\in\setNatural_0$ and $r\in\setReal^+$.
      \[
        \begin{aligned}
          0 =&\ r^2 A''_l(r) + 2r A'_l(r) \\
          &+ \boxBrackets{\roundBrackets{k^2 - U(r)}r^2 - l(l+1)} A_l(r)
        \end{aligned}
      \]
      This equation can be simplified by defining a variant of the radial function for all $l\in\setNatural_0$ and $r\in\setReal^+$.
      \[
        \function{u_l}{\setReal^+}{\setComplex}
        \separate
        u_l(r) := r A_l(r)
      \]
      Then for all $l\in\setNatural_0$ the variants $u_l$ fulfill the following differential equation for all $r\in\setReal^+$.
      \[
        u_l''(r) + \boxBrackets{k^2-U(r)-\frac{l(l+1)}{r^2}}u_l(r) = 0
      \]

      \paragraph{Differential Equation:}
      In the expansion of the final-state wave function of the scattered particle $ψ_k$ takes on the following form for all $r\in\setReal^+$ and $ϑ\in[0,π]$.
      \[
        ψ_k(r,ϑ) = \sum_{l=0}^\infty i^l (2l+1) P_l(\cos ϑ) \frac{e^{iδ_l}}{kr}g_l(r)
      \]
      Here $δ_l\in[0,2π]$ is the phase shift and $\function{g_l}{\setReal^+}{\setComplex}$ another radial function for all $l\in\setNatural_0$.
      Comparing this expression with the former wave expansion we find a proportional relation for all $l\in\setNatural_0$ and $r\in\setReal^+$.
      \[
        \frac{g_l(r)}{r} \propto A_l(r)
        \quad\implies\quad
        g_l(r) \propto u_l(r)
      \]
      For any $l\in\setNatural_0$ both differential equations for $A_l$ and for $u_l$ are linear homogeneous differential equations of second order.
      Hence, any solution of these equations multiplied by any constant is also a solution.
      As a result $g_l$ has to fulfill the same differential equation as the variants $u_l$ for all $l\in\setNatural_0$ and $r\in\setReal^+$.
      \[
        g_l''(r) + \boxBrackets{k^2-U(r)-\frac{l(l+1)}{r^2}}g_l(r) = 0
      \]

      \paragraph{Asymptotic Behavior:}
      Let $l\in\setNatural_0$ be arbitrary.
      To get a general idea of what is happening near infinity we will first look at the asymptotic behavior of the differential equation fulfilled by $g_l$.
      We will assume that $U(r)\converges[r\to\infty] 0$ which leads us to the following equation.
      Choose $r\in\setReal^+$ to be near infinity.
      \[
        g_l''(r) + k^2 g_l(r) = 0
      \]
      This is the differential equation of the classical harmonic oscillator and can be solved easily for constants $A,B\in\setComplex$.
      \[
        g_l(r) = A e^{ikr} + B e^{-ikr}
      \]
      Therefore the variants of the radial functions will asymptotically behave the same way as a spherical wave.
      To determine the values $A$ and $B$ we will use another more detailed approach based on the scattering amplitude.

      The scattering amplitude $\function{f_k}{[0,π]}{\setComplex}$ in its partial wave expansion with partial-wave amplitudes $f_{kl}\in\setComplex$ for $l\in\setNatural_0$ can be written as follows for all $ϑ\in[0,π]$.
      \[
        f_k(ϑ) = \sum_{l=0}^\infty (2l+1)f_{kl} P_l(\cos ϑ)
      \]
      Applying this formula on the wave function of the scattered particle we get the following.
      \[
        ψ_k(r) \converges
        \begin{aligned}[t]
          & \sum_{l=0}^\infty (2l+1) P_l(\cos ϑ) \\
          & \cdot \frac{1}{2ik} \boxBrackets{\roundBrackets{1+2ikf_{kl}} \frac{e^{ikr}}{r} - \frac{e^{-i(kr-lπ)}}{r}}
        \end{aligned}
      \]
      By definition, the partial-wave amplitudes are related to the phase shifts.
      \[
        1+2ikf_{kl} = e^{2iδ_l}
      \]
      Taking this equation and comparing the partial-wave expansion based on the scattering amplitude to the already given expansion we can now conclude the following.
      \[
        \begin{aligned}
          i^l &e^{iδ_l} \frac{g_l(r)}{kr}
          \longrightarrow \frac{1}{2ik} \boxBrackets{e^{2iδ_l} \frac{e^{ikr}}{r} - \frac{e^{-i(kr-lπ)}}{r}} \\
          &= \frac{e^{i\roundBrackets{δ_l+\frac{π}{2}l}}}{2ikr} \boxBrackets{e^{i\roundBrackets{kr - \frac{π}{2}l + δ_l}} - e^{-i\roundBrackets{kr - \frac{π}{2}l + δ_l}}} \\
          &=  e^{il\frac{π}{2}} \frac{e^{iδ_l}}{kr} \sin\roundBrackets{ kr -\frac{π}{2}l + δ_l } \\
          &= i^l \frac{e^{iδ_l}}{kr} \sin\roundBrackets{ kr -\frac{π}{2}l + δ_l }
        \end{aligned}
      \]
      The detailed asymptotic behavior of $g_l$ is explicitly be described below.
      \[
        g_l(r) \converges \sin\roundBrackets{ kr -\frac{π}{2}l + δ_l }
      \]

      \paragraph{Wronskian Determinant:}
      Let $V_1$ and $V_2$ be potentials with the same properties as $V$ and define $U_1$, $δ_l^{(1)}$ and $U_2$, $δ_l^{(2)}$ analog to $U$ and $δ_l$.
      Let $\function{g_l,h_l}{\setReal^+}{\setComplex}$ be radial functions such that they fulfill the following differential equations for all $r\in\setReal^+$.
      \[
        g_l''(r) + \boxBrackets{k^2-U_1(r)-\frac{l(l+1)}{r^2}}g_l(r) = 0
      \]
      \[
        h_l''(r) + \boxBrackets{k^2-U_2(r)-\frac{l(l+1)}{r^2}}h_l(r) = 0
      \]
      For twice continuously differentiable functions $\function{f,g}{\setReal^+}{\setComplex}$ we define the Wronskian determinant.
      \[
        W(f,g)\define
        \begin{vmatrix}
          f & g \\
          f' & g'
        \end{vmatrix}
        = fg' - gf'
      \]
      Then the derivative of this functional determinant is given by the following expression.
      \[
        \derivative W(f,g)
        \begin{aligned}[t]
          &= f'g' + fg'' - g'f' - gf'' \\
          &= fg'' - gf''
        \end{aligned}
      \]
      Inserting now $g_l$ and $h_l$ we can prove the proposition.
      \[
        \begin{aligned}[t]
          \derivative &W(g_l,h_l)(r) \\
          &= g_l(r) h_l''(r) - h_l(r) g_l''(r) \\
          &=
            \begin{aligned}[t]
              &g_l(r) h_l(r) \boxBrackets{U_2(r) + \frac{l(l+1)}{r^2} - k^2} \\
              &-h_l(r) g_l(r) \boxBrackets{U_1(r) + \frac{l(l+1)}{r^2} - k^2} \\
            \end{aligned}
          \\
          &= g_l(r) h_l(r) \boxBrackets{U_2(r) - U_1(r)}
        \end{aligned}
      \]
      \hfill$\square$

      \paragraph{Phase Shift Difference:}
      The integral for the phase shift difference can be computed straightforward by using the results above and the fundamental theorem of calculus.
      \[
        \begin{aligned}
          -\frac{2m}{\hbar^2 k} &\integral{0}{\infty}{g_l(r) h_l(r) \boxBrackets{V_2(r) - V_1(r)}}{r} \\
          &= -\frac{1}{k} \integral{0}{\infty}{g_l(r) h_l(r) \boxBrackets{U_2(r) - U_1(r)}}{r} \\
          &= -\frac{1}{k} \integral{0}{\infty}{\derivative W(g_l,h_l)(r)}{r} \\
          &= -\frac{1}{k} \appendValue{W(g_l,h_l)(r)}{r=0}^\infty \\
          &= -\frac{1}{k} \appendValue{\boxBrackets{g_l(r)h_l'(r) - h_l(r) g_l'(r)}}{r=0}^\infty \\
        \end{aligned}
      \]
      Because the radial functions have to be regular at the origin, we are assuming the following conditions.
      \[
        g_l(0) = h_l(0) = 0
      \]
      We already know about the asymptotic behavior of $g_l$ and $h_l$.
      Choose $r\in\setReal^+$ to be near infinity.
      \[
        g_l(r) \converges \sin\roundBrackets{ kr -\frac{π}{2}l + δ_l^{(1)} }
      \]
      \[
        h_l(r) \converges \sin\roundBrackets{ kr -\frac{π}{2}l + δ_l^{(2)} }
      \]
      Therefore we can derive the asymptotic behavior of $h_l'$ and $g_l'$ as well by computing the derivative of the asymptotic function.
      \[
        g_l'(r) \converges k \cos\roundBrackets{ kr -\frac{π}{2}l + δ_l^{(1)} }
      \]
      \[
        h_l'(r) \converges k \cos\roundBrackets{ kr -\frac{π}{2}l + δ_l^{(2)} }
      \]
      We will now insert these ideas into the calculation of the integral.
      Additionally, we use an addition theorem to finish the proof.
      \[
        \begin{aligned}
          -\frac{2m}{\hbar^2 k} &\integral{0}{\infty}{g_l(r) h_l(r) \boxBrackets{V_2(r) - V_1(r)}}{r} \\
          &= -\frac{1}{k} \lim_{r\to\infty}\boxBrackets{g_l(r)h_l'(r) - h_l(r) g_l'(r)} \\
          &= -\frac{1}{k} \lim_{r\to\infty}
            \begin{aligned}[t]
              &\left[ \sin\roundBrackets{ kr -\frac{π}{2}l + δ_l^{(1)} } \right.\\
              &\quad \cdot k \cos\roundBrackets{ kr -\frac{π}{2}l + δ_l^{(2)} } \\
              &- \sin\roundBrackets{ kr -\frac{π}{2}l + δ_l^{(2)} } \\
              &\quad \left. \cdot k \cos\roundBrackets{ kr -\frac{π}{2}l + δ_l^{(1)} } \right]
            \end{aligned}
          \\
          &= \lim_{r\to\infty}
            \begin{aligned}[t]
              &\left[ \sin\roundBrackets{ kr -\frac{π}{2}l + δ_l^{(2)} } \right.\\
              &\quad\cdot \cos\roundBrackets{ kr -\frac{π}{2}l + δ_l^{(1)} } \\
              &- \sin\roundBrackets{ kr -\frac{π}{2}l + δ_l^{(1)} } \\
              &\quad \left. \cdot \cos\roundBrackets{ kr -\frac{π}{2}l + δ_l^{(2)} } \right]
            \end{aligned}
          \\
          &= \lim_{r\to\infty} \sin
              \begin{aligned}[t]
                &\left[ \roundBrackets{kr -\frac{π}{2}l + δ_l^{(2)}} \right. \\
                &\left. - \roundBrackets{kr -\frac{π}{2}l + δ_l^{(1)}} \right]
              \end{aligned}
          \\
          &= \lim_{r\to\infty} \sin \roundBrackets{δ_l^{(2)} - δ_l^{(1)}} \\
          &= \sin \roundBrackets{δ_l^{(2)} - δ_l^{(1)}}
        \end{aligned}
      \]
      \hfill$\square$
    \end{multicols}
  % section problem_19 (end)

  \newpage

  \section*{Problem 20: Identical Particles in an Infinite Well} % (fold)
  \label{sec:problem_20}
    \begin{multicols}{2}
      \paragraph{Preliminaries:}
      Let $\mathscr{L}^2(\setReal)$ be the space of square-integrable functions with domain $\setReal$ and Lebesgue-measure λ.
      \[
        \mathscr{L}^2(\setReal) \define \set{\function{f}{\setReal}{\setComplex}}{\integral{\setReal}{}{\absolute{f}^2}{λ} < \infty}
      \]
      We define the typical scalar product for such a space as follows.
      \[
        \function{\bracket{\cdot}{\cdot}}{\mathscr{L}^2\times\mathscr{L}^2}{\setComplex}
        \separate
        \bracket{f}{g} \define \integral{\setReal}{}{\bar{f}g}{λ}
      \]
      The Hilbert space for a single particle with mass $m\in\setReal^+$ in an infinite potential well with size $a\in\setReal^+$ centered at the origin is described by $\roundBrackets{\mathscr{L}^2(\setReal), \bracket{\cdot}{\cdot}}$.
      For convenience, we define the following constants.
      \[
        ω \define \frac{π}{2a}
        \separate
        ε \define \frac{\hbar^2ω^2}{2m}
        \separate
        z \define \frac{1}{\sqrt{a}}
      \]
      With $\ket{n}\in\mathscr{L}^2(\setReal)$, we denote the eigenstates of the Hamiltonian $\function{H}{\mathscr{L}^2(\setReal)}{\mathscr{L}^2(\setReal)}$ with eigenvalues $E_n$ for all $n\in\setNatural$.
      \[
        \ket{n} \define
        \begin{cases}
          z \cos(nω\cdot) &: n=2k-1 \quad \text{for } k\in\setNatural \\
          z \sin(nω\cdot) &: n=2k \quad \text{for } k\in\setNatural
        \end{cases}
      \]
      \[
        E_n \define εn^2
      \]

      \paragraph{Hilbert Space:}
      For three identical particles the following space $\mathscr{H}$ builds a superset of the Hilbert space for these particles.
      \[
        \mathscr{H} \define \mathscr{L}^2(\setReal) \otimes \mathscr{L}^2(\setReal) \otimes \mathscr{L}^2(\setReal) \cong \mathscr{L}^2\roundBrackets{\setReal^3}
      \]
      To define the scalar product for this space we first define the product state for all $\ket{φ_1},\ket{φ_2},\ket{φ_3}\in\mathscr{L}^2(\setReal)$.
      \[
        \ket{φ_1φ_2φ_3} \define \ket{φ_1}\otimes\ket{φ_2}\otimes\ket{φ_3}
      \]
      Please note, that for convenience we will use function overloading to define the scalar product.
      \[
        \function{\bracket{\cdot}{\cdot}}{\mathscr{H}\times\mathscr{H}}{\setComplex}
      \]
      \[
        \bracket{φ_1φ_2φ_3}{ϑ_1ϑ_2ϑ_3} \define \bracket{φ_1}{ϑ_1} \bracket{φ_2}{ϑ_2} \bracket{φ_3}{ϑ_3}
      \]
      This scalar product will also be used for the Hilbert space of the three identical particles, which is possible because its a subset of $\mathscr{H}$.

      All particles are Bosons because of their spin which is equal to zero.
      Therefore $\mathscr{H}$ has to be symmetrized such that all possible wave functions are symmetric.
      We call $\roundBrackets{\mathscr{H}_\mathrm{S},\bracket{\cdot}{\cdot}}$ the Hilbert space of these three identical Bosons.
      \[
        \mathscr{H}_\mathrm{S} \define \set{\ket{φ}\in\mathscr{H}}{\forall π\in\mathrm{S}_3:\ P(π)\ket{φ}=\ket{φ}} \subset \mathscr{H}
      \]
      Here, $\mathrm{S}_3$ is the three-dimensional permutation group and $P(π)$ the related permutation operator for $π\in\mathrm{S}_3$.

      \paragraph{Eigenbasis:}
      The three particles do not interact.
      Hence, the Hamiltonian $\function{H_\mathrm{S}}{\mathscr{H}_\mathrm{S}}{\mathscr{H}_\mathrm{S}}$ for $\mathscr{H}_\mathrm{S}$ can be stated as follows.
      \[
        H_\mathrm{S} \define H\otimes\mathds{1}\otimes\mathds{1} + \mathds{1}\otimes H\otimes\mathds{1} + \mathds{1}\otimes\mathds{1}\otimes H
      \]
      We can now conclude that for all $n_1,n_2,n_3\in\setNatural$ the states $\ket{n_1n_2n_3}$ are eigenstates of $H_\mathrm{S}\vert_{\mathscr{H}}$ with eigenvalue $E_{n_1n_2n_3}$.
      \begin{align*}
        H_\mathrm{S}\ket{n_1n_2n_3}
        &= H\ket{n_1}\otimes\ket{n_2}\otimes\ket{n_3} \\
        &+ \ket{n_1}\otimes H\ket{n_2}\otimes\ket{n_3} \\
        &+ \ket{n_1}\otimes\ket{n_2}\otimes H\ket{n_3} \\
        &= \roundBrackets{E_{n_1} + E_{n_2} + E_{n_3}} \ket{n_1n_2n_3}
      \end{align*}
      \[
        E_{n_1n_2n_3} \define E_{n_1} + E_{n_2} + E_{n_3}
      \]
      Additionally, for all $m_1,m_2,m_3\in\setNatural$ the following equation shows that the states are orthonormal and consequently are building an orthonormal basis of $\mathscr{H}$.
      \begin{align*}
        &\bracket{m_1m_2m_3}{n_1n_2n_3} \\
        &= \bracket{m_1}{n_1}\bracket{m_2}{n_2}\bracket{m_3}{n_3} \\
        &= δ_{m_1n_1}δ_{m_2n_2}δ_{m_3n_3}
      \end{align*}
      But these states are in general not symmetric and may not lie in $\mathscr{H}_\mathrm{S}$ which would violate the condition that the system consists of three identical Bosons.
      Like $\mathscr{H}$ had to be symmetrized, the states also have to be symmetrized.
      For all $n_1,n_2,n_3\in\setNatural$ we define the symmetric states as follows.
      \[
        \ket{n_1n_2n_3}_\mathrm{S}\define \sum_{σ\in\mathrm{S_3}}\ket{n_{σ(1)}n_{σ(2)}n_{σ(3)}}
      \]
      Due to the linear combination and the permutation we are now making sure that the states are building a symmetric eigenbasis of $\mathscr{H}_\mathrm{S}$ with respect to $H_\mathrm{S}$.
      Of course, these symmetric states are not normalized.
      To normalize the states, we choose $n_1,n_2,n_3\in\setNatural$ to be arbitrary.
      \begin{align*}
        &\prescript{}{\mathrm{S}}{\bracket{n_1n_2n_3}{n_1n_2n_3}_\mathrm{S}} \\
        &= \sum_{π,σ\in\mathrm{S}_3} \bracket{n_{π(1)}n_{π(2)}n_{π(3)}}{n_{σ(1)}n_{σ(2)}n_{σ(3)}} \\
        &= \sum_{π,σ\in\mathrm{S}_3} δ_{n_{π(1)}n_{σ(1)}} δ_{n_{π(2)}n_{σ(2)}} δ_{n_{π(3)}n_{σ(3)}}
      \end{align*}
      We have to sum over $3!\cdot3!=36$ terms and at least six of them will be one because the permutations $σ$ and $π$ can be the same.

      \paragraph{Case $n_1\neq n_2\neq n_3 \neq n_1$:}
      Then it is clear that only these six terms can contribute to the actual sum because for every permutation $π\in\mathrm{S}_3$ we compute the result below.
      \begin{align*}
        &\sum_{σ\in\mathrm{S}_3}\bracket{n_{π(1)}n_{π(2)}n_{π(3)}}{n_{σ(1)}n_{σ(2)}n_{σ(3)}} \\
        &= \sum_{σ\in\mathrm{S}_3} δ_{{π(1)}{σ(1)}} δ_{{π(2)}{σ(2)}} δ_{{π(3)}{σ(3)}}
        = 1
      \end{align*}
      The normalization in this case is straightforward.
      \[
        \ket{n_1n_2n_3}_{\tilde{\mathrm{S}}} \define \frac{1}{\sqrt{3!}}\ket{n_1n_2n_3}_\mathrm{S}
      \]

      \paragraph{Case $n_1\neq n_2 = n_3$:}
      If two values are the same then a permutation interchanging those values will not change the actual state.
      This can be stated as follows for all permutations $π\in\mathrm{S}_3$.
      \begin{align*}
        &\sum_{σ\in\mathrm{S}_3}\bracket{n_{π(1)}n_{π(2)}n_{π(3)}}{n_{σ(1)}n_{σ(2)}n_{σ(3)}} \\
        &= \sum_{σ\in\mathrm{S}_3}
          \begin{aligned}[t]
            & \left( δ_{{π(1)}{σ(1)}} δ_{{π(2)}{σ(2)}} δ_{{π(3)}{σ(3)}} \right. \\
            & \left. + δ_{{π(1)}{σ(1)}} δ_{{π(2)}{σ(3)}} δ_{{π(3)}{σ(2)}} \right)
          \end{aligned}
        \\
        &= 2
      \end{align*}
      In this case the normalization factor changes and the normalized state is defined below.
      \[
        \ket{n_1n_2n_3}_{\tilde{\mathrm{S}}} \define \frac{1}{\sqrt{3!\cdot 2!}}\ket{n_1n_2n_3}_\mathrm{S}
      \]

      \paragraph{Case $n_1=n_2=n_3$:}
      If all of these three values are the same then every term in the sum has to be one because for every permutation $π\in\mathrm{S}_3$ we can conclude the following.
      \begin{align*}
        &\sum_{σ\in\mathrm{S}_3}\bracket{n_{π(1)}n_{π(2)}n_{π(3)}}{n_{σ(1)}n_{σ(2)}n_{σ(3)}} \\
        &= \sum_{σ\in\mathrm{S}_3}\sum_{κ\in\mathrm{S}_3} δ_{{π(1)}{σ(κ(1))}} δ_{{π(2)}{σ(κ(2))}} δ_{{π(3)}{σ(κ(3))}} \\
        &= 6
      \end{align*}
      Of course, this is a direct consequence of the following equation for all $σ\in\mathrm{S}_3$.
      \[
        \ket{n_1n_2n_3} = \ket{n_{σ(1)}n_{σ(2)}n_{σ(3)}}
      \]
      Hence, the normalization in this case can formulated as follows.
      \[
        \ket{n_1n_2n_3}_{\tilde{\mathrm{S}}} \define \frac{1}{\sqrt{3!\cdot 3!}}\ket{n_1n_2n_3}_\mathrm{S}
      \]

      \paragraph{General Normalization:}
      The results obtained here can be generalized.
      Define $p_n\in\setNatural_0$ to be the count of particles in the state $\ket{n}$ for all $n\in\setNatural$.
      Because we have three particles the following equation is fulfilled.
      \[
        \sum_{n\in\setNatural} p_n = 3
      \]
      Particles in the same state can be interchanged.
      There are $p_n!$ possible permutations for every $n\in\setNatural$ such that the state is not changed.
      Thus, for every permutation $π\in\mathrm{S}_3$ the following holds.
      \[
        \sum_{σ\in\mathrm{S}_3}\bracket{n_{π(1)}n_{π(2)}n_{π(3)}}{n_{σ(1)}n_{σ(2)}n_{σ(3)}}
        = \prod_{n\in\setNatural} p_n!
      \]
      With this we could have also defined the following general normalization rule.
      \[
        \ket{n_1n_2n_3}_{\tilde{\mathrm{S}}} \define \frac{1}{\sqrt{3!\prod_{n\in\setNatural}p_n!}} \ket{n_1n_2n_3}_\mathrm{S}
      \]
      The set $\set{\ket{n_1n_2n_3}_{\tilde{\mathrm{S}}}}{n_1,n_2,n_3\in\setNatural}$ is therefore building an orthonormal eigenbasis of $\mathscr{H}_\mathrm{S}$ with respect to $H_\mathrm{S}$.
      Please note, that the states are symmetric and therefore the order of $n_1$, $n_2$ and $n_3$ can be ignored.
      It suffices to describe the symmetric state by the numbers $(p_n)_{n\in\setNatural}$.

      \paragraph{Lowest Energy States:}
      As shown above for a normalized eigenstate $\ket{n_1n_2n_3}_{\tilde{\mathrm{S}}}$ of the Hamiltonian $H_\mathrm{S}$ with have the energy $E_{n_1n_2n_3}$ as an eigenvalue for all $n_1,n_2,n_3\in\setNatural$.
      Therefore we can find the five states with the lowest energy by direct computation.
      \begin{align*}
        &\ket{111}_{\tilde{\mathrm{S}}}: & E_{111} &= 3ε \\
        &\ket{112}_{\tilde{\mathrm{S}}}: & E_{112} &= 6ε \\
        &\ket{122}_{\tilde{\mathrm{S}}}: & E_{122} &= 9ε \\
        &\ket{113}_{\tilde{\mathrm{S}}}: & E_{113} &= 11ε \\
        &\ket{222}_{\tilde{\mathrm{S}}}: & E_{222} &= 12ε
      \end{align*}
    \end{multicols}
  % section problem_20 (end)
\end{document}