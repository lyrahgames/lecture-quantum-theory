\documentclass[10pt,fleqn]{article}
\usepackage{standard}

\geometry{left=27mm,right=27mm,top=40mm,bottom=30mm}

\newcommand{\exerciseHeader}{%
  \hrule
  \begin{center}
    \Large
    \scshape
    Advanced Quantum Theory \\ Exercise Sheet 11
  \end{center}
  \medskip
  {
    \footnotesize
    \begin{minipage}[c]{0.49\textwidth}
      Markus Pawellek \\
      markuspawellek@gmail.com
    \end{minipage}
    \hfill
    \begin{minipage}[c]{0.49\textwidth}
      \raggedleft
      \today
    \end{minipage}
  }
  \medskip
  \hrule
  \bigskip
}

\newcommand{\bra}[1]{\left\langle #1 \right\vert}
\newcommand{\ket}[1]{\left\vert #1 \right\rangle}
\newcommand{\bracket}[2]{\left\langle #1 \middle\vert #2 \right\rangle}

\begin{document}
  \selectlanguage{english}

  \exerciseHeader

  \begin{multicols}{2}
    \section*{Problem 18: Perturbation Theory} % (fold)
    \label{sec:problem_18_perturbation_theory}
      Let $\roundBrackets{\mathscr{H},\bracket{\cdot}{\cdot}}$ be a separable Hilbert space which is modeling the perturbed and the unperturbed system.
      For the formal definition of both systems we will define $\mathscr{I}$ as a countable index set with the same cardinality as an orthonormal basis of $\mathscr{H}$.

      The Hamiltonian $\function{H}{\mathscr{H}}{\mathscr{H}}$ of the unperturbed system is assumed to have a discrete spectrum of real Energies.
      \[
        σ(H) = \set{E_n}{n\in\mathscr{I}}\subset\setReal
      \]
      For all $n\in\mathscr{I}$ we denote $\ket{n}\in\mathscr{H}$ to be the unique normalized eigenstate of $H$ with the eigenvalue $E_n$.
      \[
        H\ket{n} = E_n\ket{n}
        \separate
        \bracket{n}{n} = 1
      \]
      \[
        \forall m\in\mathscr{I},m\neq n: \quad E_m\neq E_n
      \]
      Based on this and due to the self-adjointness of the Hamiltonian $H$ we can derive that
      % , for all $n,m\in\mathscr{I}$ with $n\neq m$ we can derive that $\ket{m}$ and $\ket{n}$ are orthogonal.
      % \[
      %   \bra{m}H\ket{n} = E_n \bracket{m}{n} = E_m \bracket{m}{n}
      % \]
      % \[
      %   \implies (E_n-E_m) \bracket{m}{n} = 0
      % \]
      % \[
      %   \implies \bracket{m}{n} = 0
      % \]
      % Hence, for all $n,m\in\mathscr{I}$ we can state the following.
      % \[
      %   \bracket{m}{n} = δ_{mn}
      % \]
      $H$ itself is non-degenerate and that the set $\set{\ket{n}}{n\in\mathscr{I}}$ is building an orthonormal basis of the Hilbert space $\mathscr{H}$.

      Further, we describe the perturbation of the system by the self-adjoint operator $\function{V}{\mathscr{H}}{\mathscr{H}}$.
      For the application of $V$ to $H$ we will use an interaction strength parameter which is introduced by using a function $\tilde{H}$ as shown below.
      \[
        \function{\tilde{H}}{[0,1]}{\mathrm{L}(\mathscr{H},\mathscr{H})}
        \separate
        \tilde{H}(λ) \define H + λ V
      \]
      The continuous parameterization will enable us to analyze the solution of the perturbed system $\tilde{H}(λ)$ in form of a series expansion in terms of λ.
      Please note that $\tilde{H}(1)$ is describing the system under the full perturbation.

      We will assume that the contribution of $V$ is small and therefore not changes the essential properties.
      Hence, we choose $λ\in[0,1]$ to be arbitrary.
      $\tilde{H}(λ)$ has a discrete spectrum.
       % of non-degeneracy and that of a discrete spectrum of $\tilde{H}(λ)$ for all $λ\in[0,1]$.
      % Let $n\in\mathscr{I}$ be arbitrary.
      \[
        σ\roundBrackets{\tilde{H}(λ)} = \set{E_n(λ)}{n\in\mathscr{I}} \subset \setReal
      \]
      $\tilde{H}(λ)$ is non-degenerate and we again assume its eigenstates to be normalized and therefore to build an orthonormal eigenbasis with respect to $\tilde{H}(λ)$
      \footnote{%
        The non-degeneracy of $\tilde{H}(λ)$ for all $λ\in[0,1]$ is not used explicitly.
        Therefore the derived formulas should be valid even if the perturbation described by $V$ introduces degeneracies into the non-degenerate system described by $H$.%
      }.
      So for all $n\in\mathscr{I}$ it holds that
      \[
        \tilde{H}(λ) \ket{n(λ)} = E_n(λ) \ket{n(λ)}
      \]
      \[
        \forall m\in\mathscr{I},m\neq n: \quad E_m(λ)\neq E_n(λ)
      \]
      \[
        \bracket{n(λ)}{n(λ)} = 1
      \]
      Additionally, we assume that we are able to expand $E_n(λ)$ and $\ket{n(λ)}$ in terms of $λ$ for all $λ\in[0,1]$ for some constant coefficients.
      \[
        \set{E_{nk}}{}_{k=0}^\infty \subset \setReal
        \separate
        E_n(λ) = \sum_{k=0}^\infty λ^k E_{nk}
      \]
      \[
        \set{\ket{n_k}}{}_{k=0}^\infty\subset\mathscr{H}
        \separate
        \ket{n(λ)} = \sum_{k=0}^\infty λ^k \ket{n_k}
      \]

      Taking the series expansions one can now derive inductive formulas for the coefficients.
      For that let $n\in\mathscr{I}$ and $λ\in[0,1]$ be arbitrary.
      We will start with the left-hand side of the eigenvalue equation.
      \begin{align*}
        &\tilde{H}(λ)\ket{n(λ)}\\
        &= (H+λV)\sum_{k=0}^\infty λ^k \ket{n_k} \\
        &= \sum_{k=0}^\infty λ^k H\ket{n_k} + \sum_{k=0}^\infty λ^{k+1} V\ket{n_k} \\
        &= \sum_{k=0}^\infty λ^{k} H\ket{n_k} + \sum_{k=1}^\infty λ^{k} V\ket{n_{k-1}} \\
        &= H\ket{n_0} + \sum_{k=1}^\infty λ^{k} \roundBrackets{H\ket{n_k} + V\ket{n_{k-1}}}
      \end{align*}
      For the right-hand side of the eigenvalue equation we do get the following.
      \begin{align*}
        &E_n(λ)\ket{n(λ)} \\
        &= \roundBrackets{\sum_{k=0}^\infty λ^k E_{nk}}\roundBrackets{\sum_{k=0}^\infty λ^k \ket{n_k}} \\
        &= \sum_{p,q=0}^\infty λ^{p+q} E_{np}\ket{n_q} \\
        &= \sum_{k=0}^\infty λ^k \sum_{p=0}^k E_{np}\ket{n_{k-p}}
      \end{align*}
      Looking at the eigenvalue equation as a whole we are now able to do a comparison of coefficients.
      This results in two equations.
      One for the starting values and one for all $k\in\setNatural$.
      \[
        H\ket{n_0} = E_{n0}\ket{n_0}
      \]
      \[
        H\ket{n_k} + V\ket{n_{k-1}} = \sum_{p=0}^k E_{np}\ket{n_{k-p}}
      \]
      After inserting the series expansions we can do something similar for the normalization condition.
      \begin{align*}
        1 &= \bracket{n(λ)}{n(λ)} \\
        &= \sum_{p,q=0} λ^{p+q} \bracket{n_p}{n_q} \\
        &= \sum_{k=0}^\infty λ^k \sum_{p=0}^k \bracket{n_p}{n_{k-p}}
      \end{align*}
      Again we do comparison of coefficients and get two equations.
      One for the starting value and one for all $k\in\setNatural$.
      \[
        \bracket{n_0}{n_0} = 1
      \]
      \begin{align*}
        \bracket{n_0}{n_k} + \bracket{n_k}{n_0}
        &= 2 \Re(\bracket{n_0}{n_k}) \\
        &= -\sum_{p=1}^{k-1} \bracket{n_p}{n_{k-p}}
      \end{align*}
      To make the second equation simpler consider that the overall phase is not determined in quantum mechanics.
      Hence, without loss of generality, we may assume $\bracket{n_0}{n_k}=\Re(\bracket{n_0}{n_k})\in\setReal$ is purely real.
      For all $k\in\setNatural$ the second equation becomes the following.
      \[
        \bracket{n}{n_k} = -\frac{1}{2}\sum_{p=1}^{k-1} \bracket{n_p}{n_{k-p}}
      \]

      After deriving these formulas we will first take a look at the starting value equations.
      \[
        H\ket{n_0} = E_{n0}\ket{n_0}
        \separate
        \bracket{n_0}{n_0} = 1
      \]
      We see that $\ket{n_0}$ is a normalized eigenfunction of $H$ with the eigenvalue $E_{n0}$.
      Therefore we can state the following.
      \[
        E_{n0} \in σ(H)
        \separate
        \ket{n_0} \in \set{\ket{m}}{m\in\mathscr{I}}
      \]
      We are free to choose the specific eigenfunction because this is the freedom of permuting the eigenstates.
      But for consistency we will use the straightforward definition.
      \[
        \ket{n_0} \define \ket{n}
        \separate
        E_{n0} \define E_n
      \]

      Now we take the inductive formula of the eigenvalue equation and insert the starting values to get an operator equation for $\ket{n_k}$ for all $k\in\setNatural$.
      \[
        \roundBrackets{H-E_n}\ket{n_k} = -V\ket{n_{k-1}} + \sum_{p=1}^{k} E_{np}\ket{n_{k-p}}
      \]
      But this equation gives us some information about the energy shift $E_{nk}$ as well.
      For that we will apply $\bra{n}$ on the equation for all $k\in\setNatural$.
      \begin{align*}
        0
        &= \bra{n}\roundBrackets{H-E_n}\ket{n_k} \\
        &= -\bra{n}V\ket{n_{k-1}} + \sum_{p=1}^k E_{np}\bracket{n}{n_{k-p}}
      \end{align*}
      By solving the equation for $E_{nk}$ one gets the following for all $k\in\setNatural$.
      \[
        E_{nk} = \bra{n}V\ket{n_{k-1}} - \sum_{p=1}^{k-1} E_{np}\bracket{n}{n_{k-p}}
      \]
      For the eigenstates $\ket{n_k}$ we basically have to invert the operator $H-E_n$.
      But due to its singularity this is not possible.
      For that reason we will first express $\ket{n_k}$ in terms of the orthonormal eigenbasis of $\mathscr{H}$ with respect to $H$.
      \begin{align*}
        \ket{n_k}
        &= \sum_{m\in\mathscr{I}} \bracket{m}{n_k} \ket{m} \\
        &= \bracket{n}{n_k}\ket{n} + \sum_{\substack{m\in\mathscr{I}\\m\neq n}} \bracket{m}{n_k} \ket{m}
      \end{align*}
      Let now $m\in\mathscr{I}$ with $m\neq n$ be arbitrary as well and apply $\bra{m}$ on the operator equation for $\ket{n_k}$ for all $k\in\setNatural$.
      \begin{align*}
        &\bra{m}\roundBrackets{H-E_n}\ket{n_k} \\
        &= \roundBrackets{E_m-E_n}\bracket{m}{n_k} \\
        &= -\bra{m}V\ket{n_{k-1}} + \sum_{p=1}^k E_{np}\bracket{m}{n_{k-p}} \\
        &= -\bra{m}V\ket{n_{k-1}} + \sum_{p=1}^{k-1} E_{np}\bracket{m}{n_{k-p}}
      \end{align*}
      Solving this for $\bracket{m}{n_k}$ for all $k\in\setNatural$ gives us the following equations.
      \[
        \bracket{m}{n_k} = -\frac{\bra{m}V\ket{n_{k-1}}}{E_m-E_n} + \sum_{p=1}^{k-1} \frac{E_{np}\bracket{m}{n_{k-p}}}{E_m - E_n}
      \]
      Inserting now the equations for $\bracket{n}{n_k}$ and $\bracket{m}{n_k}$ into the expansion with respect to the eigenbasis yields the following for all $k\in\setNatural$.
      \begin{align*}
        \ket{n_k} &= -\frac{1}{2}\sum_{p=1}^{k-1} \bracket{n_p}{n_{k-p}}\ket{n} \\
        &+ \sum_{\substack{m\in\mathscr{I}\\m\neq n}}
          \begin{aligned}[t]
            &\left[ -\frac{\bra{m}V\ket{n_{k-1}}}{E_m-E_n} \right. \\
            &\left. + \sum_{p=1}^{k-1} \frac{E_{np}\bracket{m}{n_{k-p}}}{E_m - E_n} \right] \ket{m}
          \end{aligned}
      \end{align*}

      With this last equation we now have obtained explicit inductive formulas with their respective starting values for $E_{nk}$ and $\ket{n_k}$ for all $k\in\setNatural$.
      Through an iterative procedure beginning by $k=1$ one can now directly obtain the formulas for the energy and state shifts.
      \[
        E_{n1} = \bra{n}V\ket{n}
      \]
      \[
        \ket{n_1} = - \sum_{\substack{m\in\mathscr{I} \\ m\neq n}} \frac{\bra{m}V\ket{n}}{E_m-E_n} \ket{m}
      \]
      The same formulas will be used for $k=2$.
      But this time we will directly insert the equations for $E_{n1}$ and $\ket{n_1}$ obtained above.
      \begin{align*}
        E_{n2}
        &= \bra{n}V\ket{n_1} - E_{n1}\bracket{n}{n_1} \\
        &= -\sum_{\substack{m\in\mathscr{I} \\ m\neq n}} \frac{\bra{m}V\ket{n}}{E_m-E_n} \bra{n}V\ket{m}
      \end{align*}
      \begin{align*}
        \ket{n_2}
        &= -\frac{1}{2} \bracket{n_1}{n_{1}}\ket{n} \\
        &+ \sum_{\substack{m\in\mathscr{I}\\m\neq n}}
          \begin{aligned}[t]
            &\left[ -\frac{\bra{m}V\ket{n_{1}}}{E_m-E_n} \right. \\
            &\left. + \frac{E_{n1}\bracket{m}{n_{1}}}{E_m - E_n} \right] \ket{m}
          \end{aligned}
        \\
        &= -\frac{1}{2} \sum_{\substack{m,k\in\mathscr{I}\\m,k\neq n}} \frac{\bra{n}V\ket{m}}{E_m-E_n} \frac{\bra{k}V\ket{n}}{E_k-E_n} \bracket{m}{k} \ket{n} \\
        &+ \sum_{\substack{m\in\mathscr{I}\\m\neq n}}
          \begin{aligned}[t]
            &\left[ \sum_{\substack{k\in\mathscr{I} \\ k\neq n}} \frac{\bra{k}V\ket{n}}{E_k-E_n} \frac{\bra{m}V\ket{k}}{E_m-E_n} \right. \\
            &\left. - \sum_{\substack{k\in\mathscr{I} \\ k\neq n}} \frac{\bra{k}V\ket{n}}{E_k-E_n} \frac{\bra{n}V\ket{n} \bracket{m}{k}}{E_m - E_n} \right] \ket{m}
          \end{aligned}
        \\
        &= -\frac{1}{2} \sum_{\substack{m\in\mathscr{I}\\m\neq n}} \frac{\absolute{\bra{m}V\ket{n}}^2}{(E_m-E_n)^2} \ket{n} \\
        &+ \sum_{\substack{m,k\in\mathscr{I}\\m,k\neq n}} \frac{\bra{k}V\ket{n} \bra{m}V\ket{k}}{(E_k-E_n)(E_m-E_n)} \ket{m} \\
        &- \sum_{\substack{m\in\mathscr{I}\\m\neq n}} \frac{\bra{n}V\ket{n} \bra{m}V\ket{n}}{(E_m-E_n)^2} \ket{m}
      \end{align*}
      This proves the proposition.
      \hfill$\square{}$
    % section problem_18_perturbation_theory (end)
  \end{multicols}

  \newpage

  \section*{Problem 20} % (fold)
  \label{sec:problem_20}
    \begin{multicols}{2}
      Let $\mathscr{L}^2(\setReal)$ be the space of square-integrable functions with domain $\setReal$ and Lebesgue-measure λ.
      \[
        \mathscr{L}^2(\setReal) \define \set{\function{f}{\setReal}{\setComplex}}{\integral{\setReal}{}{\absolute{f}^2}{λ} < \infty}
      \]
      We define the typical scalar product for such a space as follows.
      \[
        \function{\bracket{\cdot}{\cdot}}{\mathscr{L}^2\times\mathscr{L}^2}{\setComplex}
        \separate
        \bracket{f}{g} \define \integral{\setReal}{}{\bar{f}g}{λ}
      \]
      The Hilbert space for a single particle with mass $m\in\setReal^+$ in an infinite potential well with size $a\in\setReal^+$ centered at the origin is described by $\roundBrackets{\mathscr{L}^2(\setReal), \bracket{\cdot}{\cdot}}$.
      For convenience, we define the following constants.
      \[
        ω \define \frac{π}{2a}
        \separate
        ε \define \frac{\hbar^2ω^2}{2m}
        \separate
        z \define \frac{1}{\sqrt{a}}
      \]
      With $\ket{n}\in\mathscr{L}^2(\setReal)$, we denote the eigenstates of the Hamiltonian $\function{H}{\mathscr{L}^2(\setReal)}{\mathscr{L}^2(\setReal)}$ with eigenvalues $E_n$ for all $n\in\setNatural$.
      \[
        \ket{n} \define
        \begin{cases}
          z \cos(nω\cdot) &: n=2k-1 \quad \text{for } k\in\setNatural \\
          z \sin(nω\cdot) &: n=2k \quad \text{for } k\in\setNatural
        \end{cases}
      \]
      \[
        E_n \define εn^2
      \]

      For three identical particles the following space $\mathscr{H}$ builds a superset of the Hilbert space for these particles.
      \[
        \mathscr{H} \define \mathscr{L}^2(\setReal) \otimes \mathscr{L}^2(\setReal) \otimes \mathscr{L}^2(\setReal) \cong \mathscr{L}^2\roundBrackets{\setReal^3}
      \]
      To define the scalar product for this space we first define the product state for all $\ket{f},\ket{g},\ket{h}\in\mathscr{L}^2(\setReal)$.
      \[
        \ket{fgh} \define \ket{f}\otimes\ket{g}\otimes\ket{h}
      \]
      Please note, that for convenience we will use function overloading to define the scalar product.
      \[
        \function{\bracket{\cdot}{\cdot}}{\mathscr{H}\times\mathscr{H}}{\setComplex}
      \]
      \[
        \bracket{φ_1φ_2φ_3}{ϑ_1ϑ_2ϑ_3} \define \bracket{φ_1}{ϑ_1} \bracket{φ_2}{ϑ_2} \bracket{φ_3}{ϑ_3}
      \]

      All particles are Bosons because of their spin which is equal to zero.
      Therefore $\mathscr{H}$ has to be symmetrized such that all possible wave functions are symmetric.
      We call $\mathscr{H}_\mathrm{S}$ the Hilbert space of these three identical Bosons.
      \[
        \mathscr{H}_\mathrm{S} \define \set{\ket{φ}\in\mathscr{H}}{\forall π\in\mathrm{S}_3:\ P(π)\ket{φ}=\ket{φ}} \subset \mathscr{H}
      \]
      Here, $\mathrm{S}_3$ is the three-dimensional permutation group and $P(π)$ the related permutation operator for $π\in\mathrm{S}_3$.

      The three particles do not interact.
      Hence, the Hamiltonian $\function{H_\mathrm{S}}{\mathscr{H}_\mathrm{S}}{\mathscr{H}_\mathrm{S}}$ for $\mathscr{H}_\mathrm{S}$ can be stated as follows.
      \[
        H_\mathrm{S} \define H\otimes\mathds{1}\otimes\mathds{1} + \mathds{1}\otimes H\otimes\mathds{1} + \mathds{1}\otimes\mathds{1}\otimes H
      \]
      We can now conclude that for all $n_1,n_2,n_3\in\setNatural$ the states $\ket{n_1n_2n_3}$ are eigenstates of $H_\mathrm{S}\vert_{\mathscr{H}}$ with eigenvalue $E_{n_1n_2n_3}$ and are building an orthonormal basis of $\mathscr{H}$.
      \begin{align*}
        &H_\mathrm{S}\ket{n_1n_2n_3} \\
        &= H\ket{n_1}\otimes\ket{n_2}\otimes\ket{n_3} \\
        &+ \ket{n_1}\otimes H\ket{n_2}\otimes\ket{n_3} \\
        &+ \ket{n_1}\otimes\ket{n_2}\otimes H\ket{n_3} \\
        &= \roundBrackets{E_{n_1} + E_{n_2} + E_{n_3}} \ket{n_1n_2n_3}
      \end{align*}
      Additionally, for all $m_1,m_2,m_3\in\setNatural$ the following equation shows that the states are orthonormal.
      \begin{align*}
        &\bracket{m_1m_2m_3}{n_1n_2n_3} \\
        &= \bracket{m_1}{n_1}\bracket{m_2}{n_2}\bracket{m_3}{n_3} \\
        &= δ_{m_1n_1}δ_{m_2n_2}δ_{m_3n_3}
      \end{align*}
      But these states are in general not symmetric and may not lie in $\mathscr{H}_\mathrm{S}$ which would violate the condition that the system consists of three identical Bosons.
      Like $\mathscr{H}$ had to be symmetrized, the states also have to be symmetrized.
      For all $n_1,n_2,n_3\in\setNatural$ we define the symmetric states as follows.
      \[
        \ket{n_1n_2n_3}_\mathrm{S}\define \sum_{σ\in\mathrm{S_3}}\ket{n_{σ(1)}n_{σ(2)}n_{σ(3)}}
      \]
      Due to the linear combination and the permutation we are now making sure that the states are building a symmetric eigenbasis of $\mathscr{H}_\mathrm{S}$ with respect to $H_\mathrm{S}$.
      Of course, these symmetric states are not normalized.
      \begin{align*}
        &\prescript{}{\mathrm{S}}{\bracket{n_1n_2n_3}{n_1n_2n_3}_\mathrm{S}} \\
        &= \sum_{π,σ\in\mathrm{S}_3} \bracket{n_{π(1)}n_{π(2)}n_{π(3)}}{n_{σ(1)}n_{σ(2)}n_{σ(3)}}
      \end{align*}
      We have to sum over $3!\cdot3!=36$ terms and at least six of them will be one because the permutations $σ$ and $π$ can be the same.
      If $n_1\neq n_2\neq n_3 \neq n_1$ then it is clear that only these six terms can contribute to the actual sum.
      If two values are the same then we can interchange them which doubles the count of non-zero terms.
      If all of these three values are the same then every term in the sum has to be one.
      \[
        \ket{n_1n_2n_3}_{\tilde{\mathrm{S}}} \define \frac{1}{\sqrt{3!\prod_{n\in\setNatural}p_n!}} \ket{n_1n_2n_3}_\mathrm{S}
      \]

      As shown above for a normalized eigenstate $\ket{n_1n_2n_3}_{\tilde{\mathrm{S}}}$ of the Hamiltonian $H_\mathrm{S}$ with have the energy $E_{n_1n_2n_3}$ as an eigenvalue.
      Therefore we can find the five states with the lowest energy.
      \begin{align*}
        &\ket{111}_{\tilde{\mathrm{S}}}: & E_{111} &= 3ε \\
        &\ket{112}_{\tilde{\mathrm{S}}}: & E_{112} &= 6ε \\
        &\ket{122}_{\tilde{\mathrm{S}}}: & E_{122} &= 9ε \\
        &\ket{113}_{\tilde{\mathrm{S}}}: & E_{113} &= 11ε \\
        &\ket{222}_{\tilde{\mathrm{S}}}: & E_{222} &= 12ε
      \end{align*}
    \end{multicols}
  % section problem_20 (end)
\end{document}